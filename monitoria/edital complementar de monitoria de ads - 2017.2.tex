\documentclass[10pt, oneside]{memoir}
\usepackage[utf8]{inputenc}
\usepackage[T1]{fontenc}
\usepackage[portuguese]{babel}
\usepackage[final]{microtype}
\usepackage{textcomp}
\usepackage{graphicx}
\usepackage{tabu}
\usepackage{hyperref}
\usepackage[a4paper, left=30mm, top=30mm, right=20mm, bottom=20mm]{geometry}

\begin{document}
\makeoddhead{headings}{}{\textsc{Fundação Alagoana de Pesquisa, Educação e Cultura}\\\textsc{Faculdade de Tecnologia de Alagoas}}{}
\makeevenhead{headings}{}{\textsc{Fundação Alagoana de Pesquisa, Educação e Cultura}\\\textsc{Faculdade de Tecnologia de Alagoas}}{\thepage}
	
\makeevenfoot{headings}{}{}{\thepage}
\makeoddfoot{headings}{}{}{\thepage}

\setsecheadstyle{\raggedright\scshape}
\setbeforesecskip{-\onelineskip}
\setaftersecskip{\onelineskip}

\begin{center}
	Coordenação de Análise e Desenvolvimento de Sistemas\\
	Edital \textnumero~03/2017\\
	Segundo Processo Seletivo para Monitoria
\end{center}

%\vspace{30pt}

A Coordenação do Curso de Análise e Desenvolvimento de Sistemas da Faculdade de Tecnologia de Alagoas (FAT/AL), no uso das suas atribuições, consoante com o disposto no Plano Pedagógico do Curso, convoca os alunos matriculados no curso de Análise e Desenvolvimento de Sistemas para inscrição no Processo Seletivo para Monitoria no semestre letivo corrente --- 2017.2, em concordância com o § 1º do Art. 4º do Edital \textnumero~02/2017 com o propósito de preencher as vagas restantes.

\section*{Da Inscrição}
\setlength{\parindent}{0em}
\setlength{\parskip}{0.2cm}
{\raggedright
	Art. 1º. A inscrição deve ser realizada através do formulário eletrônico disponível no endereço \url{https://goo.gl/forms/ules0SI3mkKtOnmr1}, no período de 30/08/2017 à 04/09/2017.
	
	Art. 2º. Podem se inscrever os alunos regularmente matriculados no curso de Análise e Desenvolvimento de Sistemas, em apenas uma das vagas para os conjuntos de disciplinas listados no Art. 5º.
	
	Art. 3º. No ato de Inscrição o aluno deve estar aprovado nas disciplinas das quais será monitor.
	
	\section*{Das Vagas}
	
	Art. 4º. As vagas são destinadas aos cursos de Análise e Desenvolvimento de Sistemas, tendo suas vagas distribuídas conforme o Art 5º.
	
	§ 1º - Os alunos aprovados para as vagas de Monitor disponibilizadas por meio deste edital, inclusive para as vagas sem concessão de auxílio-bolsa, terão a certificação em horas complementares correspondente ao tempo disponibilizado pelo aluno para o exercício da monitoria em consonância com os Art. 11 e 13.
	
	Art. 5º. A distribuição de vagas segue conforme descrito na tabela abaixo.
	
	~
	
	\begin{tabu} to \textwidth {X[2]X[r]X[r]}
		%	\rowfont{\bfseries}
		Disciplinas&Bolsa-auxílio&Quantidade de vagas\\
		\midrule
		Introdução à Programação e Laboratório de Programação&1&2\\
		\bottomrule
	\end{tabu}
	
	~
	
	\section*{Do Processo Seletivo}
	
	Art. 6º. O processo seletivo se dá tem três etapas: Análise de Histórico Escolar, Avaliação Escrita e Oral com os professores responsáveis pelas disciplinas.
	
	Art. 7º. Será considerado aprovado o aluno que obtiver a maior pontuação ao final das etapas de seleção realizadas pelo professor ou professores responsáveis pelas disciplinas, conforme os critérios e pontuações descritos a seguir, desde que obtenha nota igual ou superior a 8,0 (oito) na Prova de Seleção.
	
	~
	
	\begin{tabu} to \textwidth {X[.1l]X[.25l]X[j]}
		%	\rowfont{\bfseries}
		Etapa&Critério&Pontuação\\
		\midrule
		1ª&Médias finais&{Somatório da pontuação equivalente ao valor da média final do aluno em cada disciplina cursada × 10. Por exemplo, se o candidato tem média final $9$ em Introdução à Programação e $8$ em Laboratório de Programação, sua pontuação neste critério será $8 \times 10 + 9 \times 10 = 170$}\\
		\midrule
		2ª&Avaliação escrita&Pontuação equivalente ao valor da Prova × 10\\
		\midrule
		3ª&Avaliação oral&Classificatória em caso de empate\\
		\bottomrule
	\end{tabu}
	
	~
	
	Art. 8º. As provas escritas serão realizadas no dia 06 de setembro de 2017 das 17:00 às 18:30 na Sala 30, localizada no Bloco A da Unidade I da FAT/AL. As provas terão duração de 1h30m. Cada candidato pode realizar prova para apenas uma disciplina. A Avaliação oral ocorrerá em segunda chamada após a divulgação das notas da prova escrita através da coordenação do curso.
	
	Art. 9º. Os conteúdos adotados para as referidas provas seguem descritos abaixo.
}
%\newpage

\textbf{Introdução à Programação e Laboratório de Programação}:
Conceitos básicos sobre a Máquina Virtual Java (JVM); Criação e configuração de projetos Java no Netbeans; Variáveis: tipos de dados primitivos; \texttt{Strings}; Entrada de dados com a classe \texttt{java.util.Scanner}; Saída com os métodos \texttt{System.out.print} e \texttt{println}; Estruturas de execução condicional; Estruturas de repetição; Funções: definição e chamada, argumentos (parâmetros) e tipos de retorno; Arrays: declaração, alocação, leitura e escrita; Interpretação e implementação de algoritmos.

\section*{Das Atividades}
\setlength{\parindent}{0em}
\setlength{\parskip}{0.2cm}
{\raggedright
	Art. 10. O aluno-monitor não terá vínculos empregatícios com a FAT.
	
	Art. 11. O aluno-monitor deverá cumprir 5 (cinco) horas semanais de atividades relacionadas à monitoria das disciplinas.
	
	Art. 12. As atividades de monitoria dar-se-ão aos sábados e/ou de segunda à sexta em horário que precede ao início das aulas regulares, para o turno noturno e para o período matutino após as 11:10.
	
	Art. 13. Ao final do período de monitoria, o aluno-monitor receberá certificado de monitoria em acordo com a disciplina objeto de suas ações e horas. Os casos omissos serão resolvidos pelo Colegiado do Curso.
	
	Art. 14. A duração da monitoria que tiver bolsa-auxílio será de 12 (doze) meses a contar da data de assinatura de contrato específico em até 15 (quinze) dias após o processo de seleção. A duração da monitoria para as vagas sem bolsa-auxílio será correspondente ao período de tempo a contar da data de aprovação até o final do semestre letivo de 2017.2
	
	Art. 15. Os professores das disciplinas listadas no Art. 5º serão responsáveis pelo planejamento e acompanhamento das atividades do aluno-monitor durante toda a duração do período de monitoria através de um Plano de Monitoria.
	
	Art. 16. O não cumprimento, o cumprimento inadequado do plano de monitoria, falta de comprometimento ou assiduidade do aluno-monitor pode resultar na exclusão do aluno-monitor do programa de monitoria e no cancelamento da bolsa-auxílio, se houver, mediante solicitação feita pelos professores responsáveis das disciplinas perate o Colegiado do curso de Análise e Desenvolvimento de Sistemas.
	
	§ 1º. Em caso de exclusão, o candidato com a maior pontuação no processo de seleção será considerado aprovado e convidado a preencher a vaga. Se não for possível preencher a vaga desta forma, um novo edital será lançado com este propósito.
	
	\section*{Da Bolsa-Auxílio}
	
	Art. 17. A bolsa-auxílio concedida ao aprovado advém do Programa de Bolsas Nacionais do Banco Santander e é celebrada através de contrato próprio nos termos do programa supracitado.
	
	Art. 18. O valor do auxílio é de R\$ 300,00 (trezentos reais) mensais por um período de até 12 (doze) meses.
	
	Art. 19. A bolsa-auxílio será concedida para o candidato aprovado com a maior pontuação.
	
	\vfill
	
	\begin{center}
		Prof. M.e Társis Wanderley Toledo\\
		Coordenador do curso de Análise e Desenvolvimento de Sistemas
			%Linguagem de Definição de Dados (DDL); Criação de Tabelas, Visões, Índices, Sequências; Restrições de Dados; Linguagem de Manipulação de Dados (DML); Linguagem de Consulta a Dados (DQL); Junção de Tabelas; Função de Agregação e Ordenação; Consultas aninhadas; Consultas aninhadas correlacionadas (EXISTS); Operações com conjuntos (UNION/EXCEPT/INTERSECT).
		%	\vspace{3em}
		Maceió, 30 de agosto de 2017
	\end{center}
	
\end{document}

